%%%%%%%%%%%%%%%%%%%%%%%%%%%%%%%%%%%%%%%%%
%
% Optics and Radar -based observations
% Assignment 4
%
%%%%%%%%%%%%%%%%%%%%%%%%%%%%%%%%%%%%%%%%%

%----------------------------------------------------------------------------------------
%	DOCUMENT CONFIGURATIONS
%----------------------------------------------------------------------------------------

\documentclass{article}

\title{\textbf {Optics and Radar Based Observations} \\ Assignment 4\\ MIT IAP Laptop Radar Lab Exercises} % Title
\def\authorivan{Ivan \v Sinkarenko}
\def\authoranu{Anuraj Rajendraprakash}
\author{\authorivan\\\authoranu}

\usepackage{graphicx}
\usepackage{fullpage}
\usepackage{url}

% load package with ``framed'' and ``numbered'' option.
\usepackage[framed,numbered,autolinebreaks,useliterate]{mcode}

\begin{document}

\maketitle % Insert the title, author and date

\centerline{Referee: Dr. Philip J. Erickson}

\setlength\parindent{0pt} % Removes all indentation from paragraphs

\renewcommand{\labelenumi}{\alph{enumi}.} % Make numbering in the enumerate environment by letter rather than number (e.g. section 6)
\clearpage

\tableofcontents

\listoffigures

\clearpage

%----------------------------------------------------------------------------------------
%	SECTION 1. Doppler Radar Processing
%----------------------------------------------------------------------------------------

\section{Doppler Radar Processing}

\subsection{Question 1}
The Doppler shift for each target is measured from the peak location of the discrete-time Fourier transform (DTFT) of the corresponding sampled waveform. Range rate is then determined by the following formula:
\begin{equation}
\label{eq:first}
\frac{f_d}{f_c}*\frac{c}{2}
\end{equation}
where $f_d$ is the Doppler shift in Hz and $f_c$ is the center frequency of the radar in Hz. Range rate calculations will obviously be limited since shifts of the peak that are greater than $\pi$ will wrap around, resulting in incorrect calculations. Adjusting the center frequency allows the range of velocities that can be correctly calculated to change. As $f_c$ is decreased, larger speeds can be calculated, but accuracy of these velocities suffers somewhat. Conversely, as $f_c$ increased, accuracy increased, but high velocities could not be detected due to aliasing. \cite{DavenportWaters:2010dsp} Figure \ref{fig:f_c_normal} shows resulting plot with initial center frequency of 2,59 GHz, Figure \ref{fig:f_c_reduced} shows the same plot with reduced center frequency of 259 MHz, whereas Figure \ref{fig:f_c_increased} has resulting plot with increased $f_c$ of 25.9 GHz.


\begin{figure}[ht]
\begin{minipage}[b]{0.33\linewidth}
\centering
\includegraphics[width=\textwidth]{Figures/f_c_normal.png}
\caption{Doppler determination with 2.59 GHz center frequency.}
\label{fig:f_c_normal}
\end{minipage}
\begin{minipage}[b]{0.33\linewidth}
\centering
\includegraphics[width=\textwidth]{Figures/f_c_reduced.png}
\caption{Doppler determination with 259 MHz center frequency.}
\label{fig:f_c_reduced}
\end{minipage}
\begin{minipage}[b]{0.33\linewidth}
\centering
\includegraphics[width=\textwidth]{Figures/f_c_increased.png}
\caption{Doppler determination with 25.9 GHz center frequency.}
\label{fig:f_c_increased}
\end{minipage}
\end{figure}

\subsection{Question 2}
\label{zpad_doppler}

\begin{figure}[ht]
\begin{minipage}[b]{0.33\linewidth}
\centering
\includegraphics[width=\textwidth]{Figures/Tp_normal.png}
\caption{Doppler determination with $Tp = 0.250 $ seconds.}
\label{fig:Tp_normal}
\end{minipage}
\begin{minipage}[b]{0.33\linewidth}
\centering
\includegraphics[width=\textwidth]{Figures/Tp_reduced.png}
\caption{Doppler determination with $Tp = 0.00250 $ seconds.}
\label{fig:Tp_reduced}
\end{minipage}
\begin{minipage}[b]{0.33\linewidth}
\centering
\includegraphics[width=\textwidth]{Figures/Tp_increased.png}
\caption{Doppler determination with $Tp = 10$ seconds.}
\label{fig:Tp_increased}
\end{minipage}
\end{figure}

The Figure \ref{fig:Tp_normal}, Figure \ref{fig:Tp_reduced} and Figure \ref{fig:Tp_increased} show the doppler vs time plots for Tp equal to 0.0025, 0.25 and 10 seconds respectively. It is clearly seen from the Figure \ref{fig:Tp_reduced} that as Tp is reduced the time resolustion and hence the range resolution becomes better but at the cost of doppler resolution. Also from Figure \ref{fig:Tp_increased} it is seen that an increase in Tp gives better doppler resolution but reduces the time i.e. the range resolution of the radar

\subsection{Question 3}

$zpad$ is zero padding, which should be bigger than the number of samples $N$. The Fast Fourier Transform (FFT) function adds zeros when this amount is bigger. The purpose of having zero padding is that having larger number of samples decreases $\Delta t$ of sampling, consequently increasing frequency resolution and allowing to study higher frequencies with the affordable frequency resolution without a threat to run into a aliasing. Figure \ref{fig:zpad_normal} shows Doppler determination plot with initial $zpad$ value of 8*N/2. Figure \ref{fig:zpad_reduced} represents the plot with zero padding reduced to the length of N. It is clearly seen that image resolution is worse on this plot. Figure \ref{fig:zpad_increased} contains Doppler determination plot with $zpad = 80*N/2$. It is seen that the actual resolution is similar to that in Figure \ref{fig:zpad_normal}. This means that zero padding has a upper limit after which bigger amount of zeros does not help anymore.

\begin{figure}[h!t]
\centering
\includegraphics[width=0.7\textwidth]{Figures/zpad_normal.png}
\caption{Doppler determination with $zpad = 8*N/2$.}
\label{fig:zpad_normal}
\end{figure}
\begin{figure}[h!t]
\centering
\includegraphics[width=0.7\textwidth]{Figures/zpad_reduced.png}
\caption{Doppler determination with $zpad = 2*N/2$.}
\label{fig:zpad_reduced}
\end{figure}
\begin{figure}[h!t]
\centering
\includegraphics[width=0.7\textwidth]{Figures/zpad_increased.png}
\caption{Doppler determination with $zpad = 80*N/2$.}
\label{fig:zpad_increased}
\end{figure}


%----------------------------------------------------------------------------------------
%	SECTION 2. Range Radar Processing
%----------------------------------------------------------------------------------------

\section{Range Radar Processing}

\subsection{Question 1}

The difference between start and stop frequency is the bandwidth. The pulse width is inversely proportional to the bandwidth. Short pulse allows in better range resolution, however, its bandwidth is small, so the reflected power is small. In practical radars it is always a compromise to select decent bandwidth and range resolution.\\
\\
A large bandwidth also reduces the dynamic range of the receiver because the noise power is proportional to the bandwidth. Also the short pulses don't have high resolution in the doppler frequency measurement. \cite{Skolnik:2001irs}\\

In Figure \ref{fig:bw_normal} the bandwidth is equal to initial bandwidth of 330 MHz. Thus, the plot has maximum range of 200 m. Figure \ref{fig:bw_reduced} contains RTI plot with bandwidth 33 MHz and Maximum range is increased by the factor of 10 accordingly. Figure \ref{fig:bw_increased} has maximum range of only 20 m, because its bandwidth is increased to 3,3 GHz.

\begin{figure}[ht]
\begin{minipage}[b]{0.33\linewidth}
\centering
\includegraphics[width=\textwidth]{Figures/bw_normal.png}
\caption{RTI with 330 MHz bandwidth.}
\label{fig:bw_normal}
\end{minipage}
\begin{minipage}[b]{0.33\linewidth}
\centering
\includegraphics[width=\textwidth]{Figures/bw_reduced.png}
\caption{RTI with 33 MHz bandwidth.}
\label{fig:bw_reduced}
\end{minipage}
\begin{minipage}[b]{0.33\linewidth}
\centering
\includegraphics[width=\textwidth]{Figures/bw_increased.png}
\caption{RTI with 3,3 GHz bandwidth.}
\label{fig:bw_increased}
\end{minipage}
\end{figure}

\subsection{Question 2}

Zero padding variable $zpad$ plays the same role as discussed in section \ref{zpad_doppler}. Figure \ref{fig:zpad_range_normal} has the RTI plot with initial zero padding of $8*N/2$, while Figure \ref{fig:zpad_range_reduced} shows the RTI plot with reduced $zpad = N$. From these two plots it is seen that plot with reduced zero padding has smoothened horizontal lines in the top and the bottom of the image, whereas in the original plot these lines have repeating curvature.

\begin{figure}[ht]
\begin{minipage}[b]{0.5\linewidth}
\centering
\includegraphics[width=\textwidth]{Figures/zpad_range_normal.png}
\caption{RTI with initial $zpad = 8*N/2$.}
\label{fig:zpad_range_normal}
\end{minipage}
\begin{minipage}[b]{0.5\linewidth}
\centering
\includegraphics[width=\textwidth]{Figures/zpad_range_reduced.png}
\caption{RTI with reduced $zpad = N$.}
\label{fig:zpad_range_reduced}
\end{minipage}
\end{figure}


\subsection{Question 3}
Two plots of RTI without and with clutter rejection are shown in Figures \ref{fig:no_rejection} and \ref{fig:rejection} respectively.
The performance improvement for clutter rejection was plotted using the following code:
\begin{lstlisting}
performance = (S2/m2)-(S(2:size(S,1),:)/m);
figure;
imagesc(R,time,performance, [-0.5, 2.5]);
colorbar;
ylabel('Performance improvment [dB]');
xlabel('range (m)');
title('Performance improvement for clutter rejection');
\end{lstlisting}

The color scale was reduced to the range of (-0.5, 2.5) to make the variations more clear. It is seen from the Figure \ref{fig:improvement} that in the areas which remain bright, espessialy the curves, on both plots there's no improvement (blueish color). This also applies for horizontal lines in the top and the bottom of the plots. The biggest improvement is on the left side of the plot and it is seen from the most yellowish color in the improvement figure. The right side of the plot is less bright saying that there's no much improvement made.

\begin{figure}[ht]
\begin{minipage}[b]{0.33\linewidth}
\centering
\includegraphics[width=\textwidth]{Figures/no_rejection.png}
\caption{RTI without clutter rejection.}
\label{fig:no_rejection}
\end{minipage}
\begin{minipage}[b]{0.33\linewidth}
\centering
\includegraphics[width=\textwidth]{Figures/rejection.png}
\caption{RTI with 2-pulse canceller clutter rejection.}
\label{fig:rejection}
\end{minipage}
\begin{minipage}[b]{0.33\linewidth}
\centering
\includegraphics[width=\textwidth]{Figures/improvement.png}
\caption{Performance improvement for clutter rejection [dB].}
\label{fig:improvement}
\end{minipage}
\end{figure}


\subsection{Question 4}

%----------------------------------------------------------------------------------------
%	SECTION 3. REFERENCES
%----------------------------------------------------------------------------------------
\newpage
\begin{thebibliography}{9}

\bibitem{Skolnik:2001irs}
Skolnik M. ~I.  (2001).
\newblock {\em Introduction to Radar Systems}.
\newblock The McGraw-Hill Companies, Inc., New York, United States.

\bibitem{DavenportWaters:2010dsp}
Mark Davenport, Drew Waters (2010).
\newblock {\em Elec 431 Digital Signal Processing}.
\newblock {\url{http://www.clear.rice.edu/elec431/projects95/radar/project.html}}.

\end{thebibliography}


\end{document}