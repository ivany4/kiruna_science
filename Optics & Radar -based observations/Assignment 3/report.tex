%%%%%%%%%%%%%%%%%%%%%%%%%%%%%%%%%%%%%%%%%
%
% Optics and Radar -based observations
% Assignment 3
%
%%%%%%%%%%%%%%%%%%%%%%%%%%%%%%%%%%%%%%%%%

%----------------------------------------------------------------------------------------
%	DOCUMENT CONFIGURATIONS
%----------------------------------------------------------------------------------------

\documentclass{article}

\title{\textbf {Optics and Radar Based Observations} \\ Assignment 3\\ Optimization of phased array antenna radiation pattern and array configuration} % Title
\def\authorivan{Ivan \v Sinkarenko}
\def\authoranu{Anuraj Rajendraprakash}
\author{\authorivan\\\authoranu}

\usepackage{graphicx}
\usepackage{fullpage}
\usepackage{url}

% load package with ``framed'' and ``numbered'' option.
\usepackage[framed,numbered,autolinebreaks,useliterate]{mcode}

\begin{document}

\maketitle % Insert the title, author and date

\centerline{Referee: Dr. Anita Enmark}

\setlength\parindent{0pt} % Removes all indentation from paragraphs

\renewcommand{\labelenumi}{\alph{enumi}.} % Make numbering in the enumerate environment by letter rather than number (e.g. section 6)
\clearpage

\tableofcontents

\listoffigures

\clearpage

%----------------------------------------------------------------------------------------
%	SECTION 1. Introduction
%----------------------------------------------------------------------------------------

\section{Introduction}
A phased array is an array of antennas in which the relative phases of the respective signals feeding the antennas are varied in such a way that the effective radiation pattern of the array is reinforced in a desired direction and suppressed in undesired directions. \cite{Wiki:2012pa}\\

\begin{figure}[h!bt]
\centering
\includegraphics[width=0.7\textwidth]{Figures/phased_array.png}
\caption{N-element receiving, parallel-feed, linear array, with equal lengths of transmission lines between each antenna element and the antenna output (at the bottom on the figure).}
\label{fig:phased_array}
\end{figure}


Consider, as in Figure \ref{fig:phased_array}, a receiving linear array made up of $N$ elements equally spaced a distance $d$ apart. The elements are assumed to be isotropic radiators in that they have uniform response for signals from all directions. Although isotropic radiators are not realizable in practice, they are convenient concept in array theory. The outputs received from all $N$ elements are summed via lines of equal length to produce a sum output voltage $E_a$. Element 1 will be taken as the reference with zero phase. From simple geometry, the difference in path length between adjacent elements for signals arriving at an angle $\theta$ with respect to the normal to the antenna, is $d\,sin\theta$. This gives a phase difference between adjacent elements of $\phi=2\pi(d/\lambda)\,sin\theta$, where $\lambda$ = wavelength of the received signal. It is assumed that there is no further amplitude or phase weighting of the received signals. For convenience, we take the amplitude of the received signal at each element to be unity. The sum of all the voltages from the individual elements, when the phase difference between adjacent elements is $\phi$ can be written
\begin{equation}
\label{eq:first}
E_a=sin\,\omega t+sin(\omega t + \phi)+sin(\omega t + 2\phi)+...+sin[\omega t +(N-1)\phi] 
\end{equation}
where $\omega$ is the angular frequency of the signal. The sum can be written
\begin{equation}
\label{eq:second}
E_a=sin\bigg[\omega t + (N-1)\frac{\phi}{2}\bigg]\,\frac{sin(N\phi/2)}{sin(\phi/2)}
\end{equation}
The first factor is a sine wave of frequency $\omega$ with a phase shift $(N-1)\phi/2$. (If the phase reference were taken at the center of the array instead at the left-hand side, this phase shift would be zero. In any event this factor is not as important as the second factor.) The second factor is and amplitude of the form $(sin\,NX)/(sin\,X)$. The magnitude of Equation \ref{eq:second} represents the field-intensity pattern, or
\begin{equation}
\label{eq:second}
|E_a(\theta)|=\bigg|\frac{sin[N\pi(d/\lambda)\,sin\,\theta]}{sin[\pi(d/\lambda)\,sin\,\theta]}\bigg|
\end{equation}
The field-intensity pattern has zeros when the numerator is zero. This occurs when $N\pi(d/\lambda)\,sin\,\theta=0,\:\pm\pi,\:\pm 2\pi,\:...,\:\pm n\pi$, where $n$ = integer. The denominator, on the other hand, is zero whenever $\pi(d/\lambda)\,sin\,\theta=0,\:\pm\pi,\:\pm 2\pi,\:...,\:\pm n\pi$. When the denominator is zero, it is seen that the numerator is also zero, and the value of $|E_a(\theta)|$ = 0/0 is indeterminate. By applying L'Hopital's rule it is found that $|E_a(\theta)|$ is a maximum and is equal to $N$ when $sin\,\theta=\pm n\lambda/d$. The maximum at $\theta$ = 0 defines the main beam of the field-intensity pattern. The other maxima are called \emph{grating lobes} and are of the same magnitude as the main beam. Grating lobes can be avoided if the spacing d between elements is equal to or less than $\lambda$. \cite{Skolnik:2001irs}

%----------------------------------------------------------------------------------------
%	SECTION 1. Discussion
%----------------------------------------------------------------------------------------

\section{Discussion}
\begin{figure}[tbh]
\centering
\includegraphics[width=\textwidth]{Figures/ratio.png}
\caption{Variation of the radiation pattern of a phased array antenna when the ratio between the wavelength and distance between individual elements varies.}
\label{fig:ratio}
\end{figure}

\begin{figure}[tbh]
\centering
\includegraphics[width=\textwidth]{Figures/elements.png}
\caption{Variation of the radiation pattern of a phased array antenna when the number of elements of the antenna varies.}
\label{fig:elements}
\end{figure}

%----------------------------------------------------------------------------------------
%	SECTION 4. CONCLUSION
%----------------------------------------------------------------------------------------
\newpage
\section{Conclusion}

%----------------------------------------------------------------------------------------
%	SECTION 5. REFERENCES
%----------------------------------------------------------------------------------------

\begin{thebibliography}{9}

\bibitem{Enmark:2012a3}
Enmark A.  (2012).
\newblock {\em Assignment 3. Optimization of phased array antenna radiation pattern and array configuration}.
\newblock Lule\aa \ University of Technology, Kiruna, Sweden.

\bibitem{Skolnik:2001irs}
Skolnik M. ~I.  (2001).
\newblock {\em Introduction to Radar Systems}.
\newblock The McGraw-Hill Companies, Inc., New York, United States.

\bibitem{Rottger:2000ip}
R\"ottger J.  (2000).
\newblock {\em The Instrumental Principles of MST Radars and Incoherent Scatter Radars and The Configuration of Radar System Hardware}.
\newblock Max Planck Institut F\"ur Aeronomie, Katlenburg-Lindau, Germany.

\bibitem{Wiki:2012pa}
Wikipedia.org. (2012).
\newblock {\em Phased array}.
\newblock {\url{http://en.wikipedia.org/wiki/Phased_array}}.

\end{thebibliography}


%----------------------------------------------------------------------------------------
%	SECTION 5. Appendix 1
%----------------------------------------------------------------------------------------
\newpage
\section{Appendix 1. Matlab code}
\lstinputlisting{assignment.m}

%----------------------------------------------------------------------------------------
%	SECTION 7. Confirmation
%----------------------------------------------------------------------------------------
\newpage
\section{Confirmation of Participation}

This is to confirm that the members of this team participated on the investigation of the required information to solve the assignment, generated their code to perform the calculation and discussed the results.\\
\vspace{2cm}
\newline
\line(1,0){200}\\
\authorivan\\
\vspace{2cm}
\newline
\line(1,0){200}\\
\authoranu\\


\end{document}