%%%%%%%%%%%%%%%%%%%%%%%%%%%%%%%%%%%%%%%%%
% Laboratory Report LaTeX Template
%
% This template has been downloaded from:
% http://www.latextemplates.com
%
% Original header:
%
% This is a LaTeX version of the sample laboratory report
% from Virginia Tech's copyrighted 08-09 CHEM 1045/1046 lab manual.
% Reproduction of this one appendix section for academic purposes
% should fall under fair use.
%
%%%%%%%%%%%%%%%%%%%%%%%%%%%%%%%%%%%%%%%%%

%----------------------------------------------------------------------------------------
%	DOCUMENT CONFIGURATIONS
%----------------------------------------------------------------------------------------

\documentclass{article}

\title{Determination of the Atomic \\ Weight of Magnesium \\ CHEM 101} % Title

\author{John \textsc{Smith}} % Author name

\begin{document}

\maketitle % Insert the title, author and date

\begin{tabular}{lr}
Date Performed: 1/1/2012 & Partner: James Smith\\ % Date the experiment was performed and partner's name
Instructor: Mary Jones % Instructor/supervisor
\end{tabular}

\setlength\parindent{0pt} % Removes all indentation from paragraphs

\renewcommand{\labelenumi}{\alph{enumi}.} % Make numbering in the enumerate environment by letter rather than number (e.g. section 6)

%----------------------------------------------------------------------------------------
%	SECTION 1
%----------------------------------------------------------------------------------------

\section{Objective}

To determine the atomic weight of magnesium via its reaction with oxygen and to study the stoichiometry of the reaction:\\

\makebox{2 Mg + O$_{2}$ $\rightarrow$ 2 Mg O}

% If you have more than one objective, uncomment the below:
%\begin{description}
%\item[First Objective] \hfill \\
%Objective 1 text
%\item[Second Objective] \hfill \\
%Objective 2 text
%\end{description}
 
%----------------------------------------------------------------------------------------
%	SECTION 2
%----------------------------------------------------------------------------------------

\section{Experimental Data}

\begin{tabular}{ll}
Mass of empty crucible & 7.28 g\\
Mass of crucible and magnesium before heating & 8.59 g\\
Mass of crucible and magnesium oxide after heating & 9.46 g\\
Balance used & \#4\\
Magnesium from sample bottle & \#1
\end{tabular}

%----------------------------------------------------------------------------------------
%	SECTION 3. Auroral Forecast
%----------------------------------------------------------------------------------------

\section{Auroral Forecast}

The purpose of this part of the assignment is to analyse data from different sensors and predict whether it is suitable conditions for seeing Aurora Borealis or not. However, by the time this report was written (April 26, 2012) it is not possible to see the northern lights because it doesn't get dark enough during the night time. For this reason, this section contains the analysis of the data taken from previous month and is compared to the images from the Auroral Large Imaging System (ALIS).
\\
The data for analysis was taken from May 28, 2012 because it was quite a spectacular Aurora that day. As a source of images and scientific data the website of Swedish Institude of Space Physics (IRF) was used. The scientific data implies magnetometer and riometer measurements.


Because of this reaction, the required ratio is the atomic weight of magnesium: 16.00g of oxygen as experimental mass of Mg: experimental mass of oxygen or $\frac{x}{1.31}=\frac{16}{0.87}$ from which, atomic weight of magnesium = 16.00 $\times \frac{1.31}{0.87}$ = 24.1 = 24 g/mol (to two significant figures).

%----------------------------------------------------------------------------------------
%	SECTION 4
%----------------------------------------------------------------------------------------

\section{Results and Conclusions}

The atomic weight of magnesium is concluded to be 24 g/mol, as determined by the stoichiometry of its chemical combination with oxygen. This result is in agreement with the accepted value.

%----------------------------------------------------------------------------------------
%	SECTION 5
%----------------------------------------------------------------------------------------

\section{Discussion of Experimental Uncertainty}

The accepted value (periodic table) is 24.3 g/mol \cite{Smith:2012qr}. The percentage discrepancy between the accepted value and the result obtained here is 1.3\%. Because only a single measurement was made, it is not possible to calculate an estimated standard deviation. \\

The most obvious source of experimental uncertainty is the limited precision of the balance. Other potential sources of experimental uncertainty are: the reaction might not be complete; if not enough time was allowed for total oxidation, less than complete oxidation of the magnesium might have, in part, reacted with nitrogen in the air (incorrect reaction); the magnesium oxide might have absorbed water from the air, and thus weigh ``too much.'' Because the result obtained is close to the accepted value it is possible that some of these experimental uncertainties have fortuitously cancelled one another.

%----------------------------------------------------------------------------------------
%	SECTION 6
%----------------------------------------------------------------------------------------

\section{Answers to Definitions}

\begin{enumerate}
\begin{item}
The \emph{atomic weight of an element} is the relative weight of one of its atoms compared to C-12 with a weight of 12.0000000$\ldots$, hydrogen with a weight of 1.008, to oxygen with a weight of 16.00. Atomic weight is also the average weight of all the atoms of that element as they occur in nature.
\end{item}
\begin{item}
The \emph{units of atomic weight} are two-fold, with an identical numerical value. They are g/mole of atoms (or just g/mol) or amu/atom.
\end{item}
\begin{item}
\emph{Percentage discrepancy} between an accepted (literature) value and an experimental value is $\frac{|\mathrm{experimental result} - \mathrm{accepted result}|}{\mathrm{accepted result}}$.
\end{item}
\end{enumerate}

%----------------------------------------------------------------------------------------
%	BIBLIOGRAPHY
%----------------------------------------------------------------------------------------

\begin{thebibliography}{9}

\bibitem{Smith:2012qr}
Smith, J.~M. and Jones, A.~B. (2012).
\newblock {\em Chemistry}.
\newblock Publisher, City, 7th edition.

\end{thebibliography}

\end{document}