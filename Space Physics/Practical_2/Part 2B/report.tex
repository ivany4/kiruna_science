%%%%%%%%%%%%%%%%%%%%%%%%%%%%%%%%%%%%%%%%%
%
% Space Physics
% Practical 2B
%
%%%%%%%%%%%%%%%%%%%%%%%%%%%%%%%%%%%%%%%%%

%----------------------------------------------------------------------------------------
%	DOCUMENT CONFIGURATIONS
%----------------------------------------------------------------------------------------

\documentclass{article}

\title{\textbf {Space Physics} \\ Practical 2B\\ Data Analysis} % Title
\def\authorivan{Ivan \v Sinkarenko}
\def\authoranu{Anuraj Rajendraprakash}
\author{\authorivan\\\authoranu}

\usepackage{graphicx}
\usepackage{fullpage}
\usepackage{url}

% load package with ``framed'' and ``numbered'' option.
\usepackage[framed,numbered,autolinebreaks,useliterate]{mcode}

\begin{document}

\maketitle % Insert the title, author and date

\centerline{Referee: Gabriella Stenberg}

\setlength\parindent{0pt} % Removes all indentation from paragraphs

\renewcommand{\labelenumi}{\alph{enumi}.} % Make numbering in the enumerate environment by letter rather than number
\clearpage

\tableofcontents

\listoffigures

\clearpage

%----------------------------------------------------------------------------------------
%	SECTION 1. Introduction
%----------------------------------------------------------------------------------------
\subsection{Question 1}
The measurment taken on 2002-03-02 at 03:29-03:30 is taken at the time when the satellite was crossing the border of magnetopause.

\begin{figure}[htb]
\centering
\includegraphics[width=0.7\textwidth]{Figures/cluster.png}
\caption{Cluster Quicklook 6-hours overview.}
\label{fig:cluster}
\end{figure}
    

%----------------------------------------------------------------------------------------
%	SECTION 2. Time Series Data
%----------------------------------------------------------------------------------------
\section{Time Series Data}

\subsection{Question 2}

Electromagnetic between 20 and 30 s. Electrostatic between 55 and 60 s. x component is facing towards the sun. It should fluctuate around zero, like the y-component does it. It happens because of the photon emission, which is captured by the probes. Thus, data correction is needed.

\begin{figure}[htb]
\centering
\includegraphics[width=\textwidth]{Figures/EFW_measurement.png}
\caption{EFW Measurement Data}
\label{fig:EFW}
\end{figure}

\begin{figure}[htb]
\centering
\includegraphics[width=\textwidth]{Figures/FGM_measurement.png}
\caption{FGM Measurement Data}
\label{fig:FGM}
\end{figure}

\begin{figure}[htb]
\centering
\includegraphics[width=\textwidth]{Figures/Staff_measurement.png}
\caption{Staff Instrument Measurement Data}
\label{fig:Staff}
\end{figure}

\subsection{Question 3}
Roughly estimated wave period is about 0.01 s (zoomed in and estimated between 2 peaks). Frequency = 1/0.01 = 100 Hz.

\subsection{Question 4}
From overview data, ion density is ~1.5 #/cm^3. 

The estimated plasma frequency from WHISPER is 16 kHz.

\subsection{Question 5}
The peaks are around 90-100 Hz. Which corresponds to our rough estimation in question 3.
%----------------------------------------------------------------------------------------
%	SECTION 3. CONCLUSION
%----------------------------------------------------------------------------------------
%\newpage
\section{Conclusion}



%----------------------------------------------------------------------------------------
%	SECTION 4. REFERENCES
%----------------------------------------------------------------------------------------
\newpage
\begin{thebibliography}{9}

\bibitem{Enmark:2012a3}
Enmark A.  (2012).
\newblock {\em Assignment 3. Optimization of phased array antenna radiation pattern and array configuration}.
\newblock Lule\aa \ University of Technology, Kiruna, Sweden.

\bibitem{Skolnik:2001irs}
Skolnik M. ~I.  (2001).
\newblock {\em Introduction to Radar Systems}.
\newblock The McGraw-Hill Companies, Inc., New York, United States.

\bibitem{Rottger:2000ip}
R\"ottger J.  (2000).
\newblock {\em The Instrumental Principles of MST Radars and Incoherent Scatter Radars and The Configuration of Radar System Hardware}.
\newblock Max Planck Institut F\"ur Aeronomie, Katlenburg-Lindau, Germany.

\bibitem{Wiki:2012pa}
Wikipedia.org. (2012).
\newblock {\em Phased array}.
\newblock {\url{http://en.wikipedia.org/wiki/Phased_array}}.

\end{thebibliography}


%----------------------------------------------------------------------------------------
%	SECTION 5. Appendix 1
%----------------------------------------------------------------------------------------
\newpage
\section{Appendix 1. Matlab code}
%\lstinputlisting{assignment.m}

\end{document}